% Created 2023-09-16 Sat 12:39
% Intended LaTeX compiler: pdflatex
\documentclass[11pt]{article}
\usepackage[utf8]{inputenc}
\usepackage[T1]{fontenc}
\usepackage{graphicx}
\usepackage{longtable}
\usepackage{wrapfig}
\usepackage{rotating}
\usepackage[normalem]{ulem}
\usepackage{amsmath}
\usepackage{amssymb}
\usepackage{capt-of}
\usepackage{hyperref}
\date{\textit{[2023-09-15 Fri 17:14]}}
\title{unit1 function of two variables}
\hypersetup{
 pdfauthor={},
 pdftitle={unit1 function of two variables},
 pdfkeywords={},
 pdfsubject={},
 pdfcreator={Emacs 29.1 (Org mode 9.6.6)}, 
 pdflang={English}}
\begin{document}

\maketitle
\tableofcontents


\section{Lecture 1 : Level Curve and Parial Derivatives}
\label{sec:org1e700a0}
Read summary \href{library/20230914T223615--lec1-level-curves-and-partial-derivatives\_\_differentiation\_mit1802.pdf}{Lec1 level curves and partial derivatives} 

Partial derivative Geometric meaning :
The partial derivative with respect to \(x\) evaluated at point
\(x_0 ,y_0\) can be approximated by :
\[
f_x(x_0,y_0) = \frac{{f(x_0 + \Delta x , y_0) - f(x_0,y_0) }}{\Delta x}
\]

So partial derivative of \(x\) measures how \(f\) changes  if we
increase \(x\) by small amount 

similarly partial derivative with respect to \(y\) at point \((x_0,y_0)\) is given by
\[
f_y = \frac{{f(x_0 , y_0 + \Delta y ) - f(x_0 , y_0) }}{\Delta x } 
\]

So partial derivative of \(y\) measures how \(f\) changes if we
increase \(y\) by small amount 

\section{Lecture 2 : Linear Approximation and Tangent Planes}
\label{sec:org4d18ee3}
Read Summary \href{library/20230915T163508--lec2-linear-approx-and-tangent-planes\_\_differentiation\_mit1802.pdf}{Lec2 linear approx and tangent planes}
\end{document}